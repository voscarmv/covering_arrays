%% LyX 2.0.3 created this file.  For more info, see http://www.lyx.org/.
%% Do not edit unless you really know what you are doing.
\documentclass[spanish]{article}
\usepackage[T1]{fontenc}
\usepackage[latin9]{inputenc}
\usepackage{listings}
\usepackage{amstext}

\makeatletter
%%%%%%%%%%%%%%%%%%%%%%%%%%%%%% Textclass specific LaTeX commands.
\newcommand{\lyxaddress}[1]{
\par {\raggedright #1
\vspace{1.4em}
\noindent\par}
}

\makeatother

\usepackage{babel}
\addto\shorthandsspanish{\spanishdeactivate{~<>.}}

\begin{document}

\title{Implementaci�n del c�mputo del un-ranking dada una sumatoria de $n$
coeficientes binomiales que tiene como resultado un valor $N$}


\author{Ing. Vicente Oscar Mier Vela}


\date{10 de julio del 2013}

\maketitle

\lyxaddress{Curso proped�utico del 2013, a cargo del Dr Jose Torres-Jimenez,
CINVESTAV, UNIDAD TAMAULIPAS, LABORATORIO DE TECNOLOG�AS DE INFORMACI�N,
Parque Cient�fico y Tecnol�gico TECNOTAM -- Km. 5.5 carretera Cd.
Victoria-Soto La Marina C.P. 87130 Cd. Victoria, Tamps. Tel�fono:
(834) 107 02 20 -- Fax: (834) 107 02 24 y (834) 314 73 92, vinculacion@tamps.cinvestav.mx}
\begin{abstract}
Estimaci�n del polinomio mayor-qu� correspondiente a una sumatoria
de coeficientes binomiales por medio de aproximaciones num�ricas en
C.
\end{abstract}

\section{Introducci�n}

Es posible aproximarnos a los valores de $a_{i}$ en la siguiente
expresi�n:

\[
N\text{=\ensuremath{\left(_{1}^{a_{1}}\right)}+\ensuremath{\left(_{2}^{a_{2}}\right)}+...+\ensuremath{\left(_{n}^{a_{n}}\right)}}
\]


Donde los valores de $a_{i}$ deben satisfacer la siguiente condici�n:

\[
a_{1}<a_{2}<...<a_{n}
\]


Estos valores representan el poliniomio mayor-qu� correspondiente
a un valor $N$ con una cantidad $n$ de coeficientes binomiales.

Podemos aproximarnos al valor de cualquier coeficiente binomial de
la sumatoria anterior utilizando la siguiente expresi�n:

\[
\left(_{k}^{n}\right)\approx\frac{n^{k}}{k!}
\]


Si conocemos el resultado de $\left(_{k}^{n}\right)$ (llam�mosle
$C$), es posible despejar el valor de $n$, de esta manera:

\[
n=\sqrt[k]{k!\times C}
\]


El coeficiente binomial $\left(_{n}^{a_{n}}\right)$ es el de valor
m�s alto en la sumatoria. Podemos aproximarnos al valor de $a_{n}$
por medio de la ecuaci�n anterior. Partiendo de esta aproximaci�n,
podemos proceder a probar con valores cada vez m�s grandes de $a_{n}$,
de tal manera que $\left(_{n}^{a_{n}}\right)$ no exceda el valor
de $N$.

Una vez que obtenemos el valor de $\left(_{n}^{a_{n}}\right)$ m�s
cercano a $N$, le restamos $\left(_{n}^{a_{n}}\right)$ a $N$ y
continuamos con $a_{n-1}$, de la misma manera que en el paso anterior,
pero utilizando el resultado de $N\text{-\ensuremath{\left(_{n}^{a_{n}}\right)}}$
como la ``nueva$N$''. Este proceso se repite hasta llegar a $a_{1}$.


\section{Implementaci�n en C}

El siguiente programa realiza la operaci�n discutida en la introducci�n.
Utiliza varias funciones de reportes anteriores, como la funci�n coef\_bin2(),
y nroot(), para auxiliar en el c�mputo del polinomio mayor-qu�:

\lstinputlisting[basicstyle={\ttfamily}]{unrank.c}

Para el c�mputo de las ra�ces, se utiliz� el algoritmo desarrollado
en el CINVESTAV para computar las $\eta$�simas ra�ces de un valor
dado\cite{key-3}.


\section{Ejemplo de ejecuci�n}

Para el comando

\lstinputlisting[basicstyle={\ttfamily}]{comm.sh}

el programa arroja el siguiente resultado. La �ltima l�nea de la salida
representa el polinomio mayor-qu� correspondiente a los valores $n\text{=7, N=300}$

\lstinputlisting[basicstyle={\ttfamily}]{out.log}

Se puede confirmar que dicho polinomio es correcto usando el programa
rank, de un reporte anterior:

\lstinputlisting[basicstyle={\ttfamily}]{comm2.sh}

Aqu�, las comillas invertidas sustituyen al comando unrank 7 300 por
su salida (el polinomio $\left\{ 0,2,4,7,8,9,10\right\} $). Este
comando arroja el resultado:

\lstinputlisting[basicstyle={\ttfamily}]{out2.log}

que es el valor de $N$.
\begin{thebibliography}{1}
\bibitem{key-3} http://dx.doi.org/10.1080/00207160.2010.528755\end{thebibliography}

\end{document}
