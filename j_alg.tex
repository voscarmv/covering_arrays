%% LyX 2.0.3 created this file.  For more info, see http://www.lyx.org/.
%% Do not edit unless you really know what you are doing.
\documentclass[spanish]{article}
\usepackage[T1]{fontenc}
\usepackage[latin9]{inputenc}
\usepackage{listings}
\usepackage{amstext}
\usepackage{babel}
\addto\shorthandsspanish{\spanishdeactivate{~<>.}}

\begin{document}

\title{Implementaci�n del Algoritmo J en C}


\author{Ing. Vicente Oscar Mier Vela}

\maketitle
El siguiente programa en C recibe un arreglo de valores num�ricos
para asignarle valores constantes a cada una de las variables de la
siguiente sumatoria:

\[
\sum_{i\text{=1}}^{n}A_{1}i^{1}\text{+}A_{2}i^{2}+...+A_{\delta}i^{\delta}
\]


En el programa, el primer argumento representa el valor de la variable
$n$, y los siguientes argumentos representan, en orden, los valores
de los coeficientes $A_{1},A_{2},...,A_{\delta}$

\lstinputlisting[basicstyle={\ttfamily}]{j_alg.c}

El programa se compila usando gcc (Debian 4.7.2-5) 4.7.2 de la siguiente
manera:

\lstinputlisting[basicstyle={\ttfamily}]{com3.sh}

Al ejecutar el comando

\lstinputlisting[basicstyle={\ttfamily}]{com2.sh}

se obtiene el resultado de la siguiente sumatoria:

\[
\sum_{i\text{=1}}^{3}i\text{+}2i^{2}-2i^{3}
\]


El programa arroja la siguiente salida en stdout:

\lstinputlisting[basicstyle={\ttfamily}]{com1.sh}
\end{document}
