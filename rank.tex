%% LyX 2.0.3 created this file.  For more info, see http://www.lyx.org/.
%% Do not edit unless you really know what you are doing.
\documentclass[spanish]{article}
\usepackage[T1]{fontenc}
\usepackage[latin9]{inputenc}
\usepackage{listings}
\usepackage{amstext}

\makeatletter
%%%%%%%%%%%%%%%%%%%%%%%%%%%%%% Textclass specific LaTeX commands.
\newcommand{\lyxaddress}[1]{
\par {\raggedright #1
\vspace{1.4em}
\noindent\par}
}

\makeatother

\usepackage{babel}
\addto\shorthandsspanish{\spanishdeactivate{~<>.}}

\begin{document}

\title{Implementaci�n del c�mputo del ranking de un polinomio mayor-qu�}


\author{Ing. Vicente Oscar Mier Vela}


\date{10 de julio del 2013}

\maketitle

\lyxaddress{Curso proped�utico del 2013, a cargo del Dr Jose Torres-Jimenez,
CINVESTAV, UNIDAD TAMAULIPAS, LABORATORIO DE TECNOLOG�AS DE INFORMACI�N,
Parque Cient�fico y Tecnol�gico TECNOTAM -- Km. 5.5 carretera Cd.
Victoria-Soto La Marina C.P. 87130 Cd. Victoria, Tamps. Tel�fono:
(834) 107 02 20 -- Fax: (834) 107 02 24 y (834) 314 73 92, vinculacion@tamps.cinvestav.mx}
\begin{abstract}
C�lculo del ranking de un polinomio mayor-qu� en C, utilizando un
bucle que suma los valores de una secuencia de coeficientes binomiales.
\end{abstract}

\section{Introducci�n}

Para computar el ranking de un polinomio, se emplea la siguiente f�rmula:

\[
\sum_{i\text{=0}}^{k-1}\left(_{i\text{+1}}^{P_{i}}\right)
\]


La expresi�n $\left(_{i\text{+1}}^{P_{i}}\right)$ representa el coeficiente
binomial de $P_{i}$ en $i\text{+1}$, y se calcula de la siguiente
manera \cite{key-3}:

\[
\left(_{k}^{n}\right)\text{=}\frac{n!}{(n-k)!k!}
\]


Una forma alternativa de calcular este valor, es la siguiente:

\[
\left(_{k}^{n}\right)\text{=}\frac{n}{1}\times\frac{n\text{-1}}{2}\times...\times\frac{n-\left(k-1\right)}{k}
\]


El siguiente programa en C implementa este �ltimo m�todo para calcular
los valores de los coeficientes binomiales que componen la sumatoria
mencionada al principio. Esto elimina la necesidad de utilizar la
funci�n de factoreo para calcular los valores de los coeficientes
binomiales.


\section{Implementaci�n en C}

El programa simplemente recibe una lista ordenada de valores $P_{i}$
y procede a computar la sumatoria de sus coeficientes binomiales con
valores consecutivos de $k$ por medio de un bucle que llama a la
funci�n coef\_bin2():

\lstinputlisting[basicstyle={\ttfamily}]{rank.c}


\section{Ejemplo de ejecuci�n}

Para el comando

\lstinputlisting[basicstyle={\ttfamily}]{comm.sh}

el programa simplemente arroja

\lstinputlisting[basicstyle={\ttfamily}]{out.log}
\begin{thebibliography}{1}
\bibitem{key-3} Discrete Mathematics, Fourth Edition, Kenneth A.
Ross, Charles R. Wright, 1999, p. 271\end{thebibliography}

\end{document}
