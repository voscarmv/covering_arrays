%% LyX 2.0.3 created this file.  For more info, see http://www.lyx.org/.
%% Do not edit unless you really know what you are doing.
\documentclass[spanish]{article}
\usepackage[T1]{fontenc}
\usepackage[latin9]{inputenc}
\usepackage{listings}

\makeatletter
%%%%%%%%%%%%%%%%%%%%%%%%%%%%%% Textclass specific LaTeX commands.
\newcommand{\lyxaddress}[1]{
\par {\raggedright #1
\vspace{1.4em}
\noindent\par}
}

\makeatother

\usepackage{babel}
\addto\shorthandsspanish{\spanishdeactivate{~<>.}}

\begin{document}

\title{Ranking y unranking de �rboles}


\author{Ing. Vicente Oscar Mier Vela}


\date{11 de julio del 2013}

\maketitle

\lyxaddress{Curso proped�utico del 2013, a cargo del Dr Jose Torres-Jimenez,
CINVESTAV, UNIDAD TAMAULIPAS, LABORATORIO DE TECNOLOG�AS DE INFORMACI�N,
Parque Cient�fico y Tecnol�gico TECNOTAM -- Km. 5.5 carretera Cd.
Victoria-Soto La Marina C.P. 87130 Cd. Victoria, Tamps. Tel�fono:
(834) 107 02 20 -- Fax: (834) 107 02 24 y (834) 314 73 92, vinculacion@tamps.cinvestav.mx}
\begin{abstract}
Ranking de �rboles utilizando el algoritmo de Pr�ffer junto con la
regla de Rufini. Unranking utilizando la operaci�n inversa del algoritmo
de Pr�ffer y la operaci�n inversa de la regla de Rufini.
\end{abstract}

\section{Introducci�n}

Por medio de la aplicaci�n del algoritmo de Pr�ffer \cite{key-3}\cite{key-1}
es posible asignarle un polinomio �nico a cada �rbol posible. Posteriormente,
a partir de este polinomio, es posible derivar un �nico n�mero decimal,
correspondiente al �rbol en cuesti�n, obteniendo el residuo de la
regla de Ruffini \cite{key-2} para dicho polinomio de Pr�ffer.


\section{Implementaci�n en C}

La implementaci�n consiste en dos funciones: prufer\_ruffini(), para
obtener el n�mero decimal �nico para un �rbol espec�fico, a partir
de la lista de sus arcos y la cantidad de nodos; y unruffini\_unpruffer(),
para obtener dicha lista de arcos a partir del n�mero de Ruffini del
�rbol y su n�mero de arcos.

A continuaci�n se muestra el c�digo fuente de la funci�n prufer\_ruffini():

\lstinputlisting[basicstyle={\ttfamily}]{pr.c}

Y aqu�, el c�digo fuente de la funci�n unruffini\_unpruffer():

\lstinputlisting[basicstyle={\ttfamily}]{rp.c}


\section{Ejemplo de ejecuci�n}

La ejecicuci�n de la funci�n prufer\_ruffini(), como se muestra en
su c�digo fuente, produce la siguiente salida:

\lstinputlisting[basicstyle={\ttfamily}]{out1.log}

Mientras que la ejecuci�n de unruffini\_unpruffer(), como se muestra
en su respectivo c�digo, produce:

\lstinputlisting[basicstyle={\ttfamily}]{out2.log}
\begin{thebibliography}{1}
\bibitem{key-3} http://www.proofwiki.org/wiki/Labeled\_Tree\_from\_Pr\%C3\%BCfer\_Sequence

\bibitem{key-1} http://en.wikipedia.org/wiki/Pr\%C3\%BCfer\_sequence

\bibitem{key-2} http://en.wikipedia.org/wiki/Ruffini\%27s\_rule\end{thebibliography}

\end{document}
