%% LyX 2.0.3 created this file.  For more info, see http://www.lyx.org/.
%% Do not edit unless you really know what you are doing.
\documentclass[spanish]{article}
\usepackage[T1]{fontenc}
\usepackage[latin9]{inputenc}
\usepackage{listings}

\makeatletter
%%%%%%%%%%%%%%%%%%%%%%%%%%%%%% Textclass specific LaTeX commands.
\newcommand{\lyxaddress}[1]{
\par {\raggedright #1
\vspace{1.4em}
\noindent\par}
}

\makeatother

\usepackage{babel}
\addto\shorthandsspanish{\spanishdeactivate{~<>.}}

\begin{document}

\title{Implementaci�n del algoritmo para computar ra�ces $\eta$�simas utilizando
restas}


\author{Ing. Vicente Oscar Mier Vela}


\date{10 de julio del 2013}

\maketitle

\lyxaddress{Curso proped�utico del 2013, a cargo del Dr Jose Torres-Jimenez,
CINVESTAV, UNIDAD TAMAULIPAS, LABORATORIO DE TECNOLOG�AS DE INFORMACI�N,
Parque Cient�fico y Tecnol�gico TECNOTAM -- Km. 5.5 carretera Cd.
Victoria-Soto La Marina C.P. 87130 Cd. Victoria, Tamps. Tel�fono:
(834) 107 02 20 -- Fax: (834) 107 02 24 y (834) 314 73 92, vinculacion@tamps.cinvestav.mx}
\begin{abstract}
Introducci�n al m�todo desarrollado en el CINVESTAV para computar
ra�ces utilizando una tabulaci�n especial y restas. Ilustraci�n de
una implementaci�n del mismo en lenguae C.
\end{abstract}

\section{Introducci�n}

El algoritmo en cuesti�n \cite{key-3}, desarrollado en el CINVESTAV
por parte del Dr. Torres et al, computa las ra�ces $\eta$�simas de
un valor constate dado, por medio de una tabla de la forma:

\lstinputlisting[basicstyle={\ttfamily}]{tabla1.txt}

Para obtener la ra�z $\eta$�sima de un valor constante, se selecciona
la $\eta$�sima fila de la tabla, y se realiza el procedimiento que
se ilustra a continuaci�n por medio del siguiente c�digo fuente.


\section{Implementaci�n en C}

Este programa genera una tabla, lo suficientemente grande para albergar
la $\eta$�sima fila, y as� poder calcular la ra�z $\eta$�sima de
un valor x dado a la funci�n nroot():

\lstinputlisting[basicstyle={\ttfamily}]{nroot.c}


\section{Ejemplo de ejecuci�n}

El programa acepta dos argumentos de entrada: el primero es el grado
de la ra�z ($\eta$) mientras que el segundo es el valor del cual
se va a obtener la ra�z. Despu�s, se genera la matriz anteriormente
mencionada, para el caso espec�fico que se requiera, segun los argumentos
de entrada. Posteriormente, se despliega dicha matriz, y se utiliza
el �ltimo rengl�n de la misma para computar la ra�z. La salida del
programa para

\lstinputlisting[basicstyle={\ttfamily}]{comm.sh}

es la siguiente

\lstinputlisting[basicstyle={\ttfamily}]{out.log}
\begin{thebibliography}{1}
\bibitem{key-3} http://dx.doi.org/10.1080/00207160.2010.528755\end{thebibliography}

\end{document}
